%- HandOut Flag -----------------------------------------------------------------------------------------
\makeatletter
\@ifundefined{ifHandout}{%
  \expandafter\newif\csname ifHandout\endcsname
}{}
\makeatother

%- D0cum3nt ----------------------------------------------------------------------------------------------
\documentclass[beamer,10pt]{standalone}   
%\documentclass[beamer,10pt,handout]{standalone}  \Handouttrue  

\ifHandout
	\setbeameroption{show notes} %print notes   
\fi

	
%- Packages ----------------------------------------------------------------------------------------------
\usepackage{custom-style}
\usetikzlibrary{positioning}
\usepackage{multicol}


%--Beamer Style-----------------------------------------------------------------------------------------------
\usetheme{toninus}
\usepackage{animate}
\usetikzlibrary{positioning, arrows}
\usetikzlibrary{shapes}

\begin{document}


%-------------------------------------------------------------------------------------------------------------------------------------------------
\subsection{Homotopy comomentum maps}\label{frame:hcmm-main}
\begin{frame}[fragile]{Homotopy comomentum maps}
	Consider a Lie algebra action $v:\mathfrak{g} \to \mathfrak{X}(M)$  \underline{preserving the $n$-plectic form $\omega$}.
	\vfill
	\begin{defblock}[Homotopy comomentum map \emph{(Callies, Fregier, Rogers, Zambon)}]
		\ifHandout
			\includestandalone{Pictures/Figure_Lifting}
		\else
			\includestandalone{Pictures/Frame_Lifting}
		\fi					
	\end{defblock}
	\onslide<4->{
	\begin{lemblock}[HCMM unfolded  (CFRZ16)]
			%
			HCMM is a sequence of (graded-skew) multilinear maps:
			\begin{displaymath}
				(f)  = \big\lbrace f_k: \; \Lambda^k{\mathfrak g} \to L^{1-k} \subseteq \Omega^{n-k}(M) 
				~\big\vert~ 0\leq k \leq n+1  \big\rbrace
			\end{displaymath}
			\emph{fulfilling:}%\emph{such that:}
			\begin{itemize}
				\item<5-> $f_0 = 0 $, $f_{n+1} = 0$
				\item<6-> $d f_k (p) = f_{k-1} (
				\tikz[baseline,remember picture]{\node[rounded corners,
                        fill=green!5,draw=green!30,anchor=base]            
            			(target) {$\partial $ };
            	}				
				p)  - (-1)^{\frac{k(k+1)}{2}} \iota(v_p) \omega 
				\qquad\scriptstyle \forall p \in \Lambda^k(\mathfrak{g}),\; \forall k=1,\dots n+1$
			\end{itemize}
		\onslide<7->{
			\tikz[overlay,remember picture]
			{
				\node[rounded corners,
	                 draw=green!30,anchor=base]            
	            	 (base) at ($(current page.east)-(3,3)$) [rotate=-0,align=center] {\footnotesize{\hyperlink{frame:CE-complex}{\emph{Chevalley-Eilenberg boundary op.}}}};
			}	
		\begin{tikzpicture}[overlay,remember picture]
	    	\path[->] (base.west) edge[bend right,green](target.north east);
	    \end{tikzpicture}
	    }
	\end{lemblock}	
	}
	\vfill
\end{frame}
\note[itemize]{
	\item  An infinitesimal symmetry is a lie algebra morphism such that $\mathcal{L}_{v_x} \omega = 0 ~ \forall x \in \mathfrak{g}$.
	\\ (It is also call an infinitesimal multisymplectic action and $v_x$ is the infinitesimal generator of the action, corresponding to $x \in \mathfrak g$.) 
	\item Essentially, admitting a comoment maps mean that $v$ acts via Hamiltonian vector fields.
	\item In mechanical terms, a moment map is a tool associated with a Hamiltonian action of a Lie group on a symplectic manifold, used to construct conserved quantities for the action.(see \ref{frame:HCMMandConserved} in appendix.
}
%-------------------------------------------------------------------------------------------------------------------------------------------------




\end{document}