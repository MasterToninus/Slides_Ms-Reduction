%- HandOut Flag -----------------------------------------------------------------------------------------
\makeatletter
\@ifundefined{ifHandout}{%
  \expandafter\newif\csname ifHandout\endcsname
}{}
\makeatother

%- D0cum3nt ----------------------------------------------------------------------------------------------
%\documentclass[beamer,10pt]{standalone}   
\documentclass[beamer,10pt,handout]{standalone}  \Handouttrue  

\ifHandout
	\setbeameroption{show notes} %print notes   
\fi

	
%- Packages ----------------------------------------------------------------------------------------------
\usepackage{custom-style}
\usepackage{math}
\usetikzlibrary{positioning}
\usepackage{multicol}


%--Beamer Style-----------------------------------------------------------------------------------------------
\usetheme{toninus}
\usepackage{animate}
\usetikzlibrary{positioning, arrows}
\usetikzlibrary{shapes}

\renewcommand{\action}{\curvearrowright} 
\makeatletter
\def\blfootnote{\gdef\@thefnmark{}\@footnotetext}
\makeatother

\begin{document}

%-------------------------------------------------------------------------------------------------------------------------------------------------
\subsection{Symplectic singular reduction}
\begin{frame}{Singular reduction schemes}
 \begin{block}[The gist of...]
 
 \end{block}
 
 \begin{thmblock}[S--W reduction]
 
 \end{thmblock}

\end{frame}
\note[itemize]{
 \item
}
%-------------------------------------------------------------------------------------------------------------------------------------------------

%-------------------------------------------------------------------------------------------------------------------------------------------------
\begin{frame}[shrink]{Smooth objects on a singular set}
	Consider $N$ closed subset of $M$.
	\vfill
	\pause
	\begin{defblock}
	 $I_N$ = ideal of smooth functions vanishing over $N$.
	\end{defblock}
	\vfill
	\pause

	\begin{columns}[T]
		\setlength{\belowdisplayskip}{5pt}
		\begin{column}{.65\linewidth}
			%
			\centering \it
				\begin{defblock}[v.f tangent to $N$]
					\begin{displaymath}
						\X_N(M):=
						\left\lbrace
							v \in \X(M)
						~\Big\vert~
							\L_v(I_N) \subseteq I_N
						\right\rbrace
					\end{displaymath}
				\end{defblock}
				\begin{defblock}[v.f vanishing on $N$]
					\begin{displaymath}
						I_\X(N):=
						\left\lbrace
							v \in \X(M)
						~\Big\vert~
							\L_v(C^\infty(M)) \subseteq I_N
						\right\rbrace
					\end{displaymath}				
				\end{defblock}				
		\end{column}	
		%
		%
		\begin{column}{.35\linewidth}
			\centering 
			\includestandalone[width=0.9\textwidth]{Pictures/Figure_VfTangentN}			
		\end{column}	
	\end{columns}			
	%	

				%

				%	
		\begin{tcolorbox}[
		enhanced,frame hidden,borderline={0.5pt}{0pt}{blue},
		arc=5pt,colback=white,
		colbacktitle=white,]
			\color{blue}{\textbf{Lem}:} If $N$ is smoothly embedded,  $\X(N)\cong \dfrac{X_N(M)}{I_\X}$.
		\end{tcolorbox}


		\begin{defblock}[Differential form vanishing on $N$]
			\begin{displaymath}
				I_{\Omega(N)}:=
				\left\lbrace
					\alpha\in\Omega^k(M
				~\left\vert~
					\begin{array}{l r}
						k\geq 0,~		\\		
						\alpha(u_1,\ldots,u_k) \in I_N & \forall u_i \in\X_N(M)
					\end{array}
				\right\rbrace\right.
			\end{displaymath}
		\end{defblock}

\end{frame}
\note[itemize]{
 \item $N$ is not a submanifold in general. An example is $N=\mu^{-1}(0)$ for a certain smooth map $N$.
 \item observe that if $N$ is a smooth embedded submanifold, one has that $\X(N)\cong \frac{X_N(M)}{I_\X}$  

}
%-------------------------------------------------------------------------------------------------------------------------------------------------



%-------------------------------------------------------------------------------------------------------------------------------------------------
\subsection{Multisymplectic singular reduction}
\begin{frame}{Reducible smooth objects}
	Consider $\g \action M$ by vector field tangent to $N$ \hfill($\underline{\g}\subseteq \X_N(M)$)
	
	\begin{defblock}[Reducible v.fields w.r.t. $\g\action M$]
			\begin{displaymath}
				\X(M)_{[N]} :=
				\left\lbrace
					v \in \X(M)
				~\left\vert~
					\begin{array}{l}
						\L_v (I_N) \subseteq I_N	\\		
						\L_v (\X_g) \subseteq \X_g + I_\X
					\end{array}
				\right\rbrace\right.
			\end{displaymath}
			\blfootnote{
			 $\X_g$ = $C^\infty(M)$-module generated by the fundamental distribution.
			}	

	\end{defblock}	
	
	\begin{defblock}[Reducible forms w.r.t. $\g\action M$]
		\begin{displaymath}
			\Omega(M)_{[N]} :=
			\left\lbrace
				\alpha \in \Omega(M)
			~\left\vert~
				\begin{array}{l r}
					\L_\underline{\xi} \,\alpha \in I_{\Omega(N)}	\\		
					\iota_\underline{\xi} \,\alpha \in I_{\Omega(N)}	& \forall \xi \in \g				\end{array}
			\right\rbrace\right.
		\end{displaymath}	
	\end{defblock}	

	\begin{defblock}[Reducible Hamiltonian forms w.r.t. $\g\action M$]
		\begin{displaymath}
			(\Omega(M)_{ham}^{n-1})_{[N]} :=
			\left\lbrace
				\alpha \in \Omega(M)_{ham}^{n-1}
			~\left\vert~
				\begin{array}{l r}
					\alpha \text{ is a reducible form} \\
					\vHam_\alpha \text{ is a reducible v.field}
				\end{array}
			\right\rbrace\right.
		\end{displaymath}	
	\end{defblock}		

	
\end{frame}
\note[itemize]{
 \item $\g\action M$ by vector field tangent to $N$ means that $\underline{\xi}\in\X_n(M) \forall \xi \in \g$
 \item Spelling out the definition: reducible vector fields are 
 \\i) v.f. tangent to N 
 \\ii) such that their commutator with any $\underline{\xi}$ lies in $\X_\g$ along $N$.
 \item more algebraically, they stabilize both $I_N$ and $\X_g + I_\X$.
 \item observe that $\L_v I_\X \subseteq I_\X$ since , $\forall u \in I_\X$ $\forall f \in C^\infty(M)$ one has\\
 $(\L_v u) f = \L_{[v,u]} f = \L_v\L_u f - \L_u\L_v f \in I_N$
 \item
}
%-------------------------------------------------------------------------------------------------------------------------------------------------

%-------------------------------------------------------------------------------------------------------------------------------------------------
\begin{frame}{Reducible observables}
	\begin{defpropblock}[Reducible $L_\infty$-observables]
		Is the {\color{blue!70!black}$L_\infty$-subalgebra} of $L_\infty(M,\omega)$ given by
		\begin{displaymath}
			L_\infty(M,\omega)_{[N]}^k :=
			\begin{cases}
				\Omega^{n-1-k}(M)_{[N]} 
				\qquad\text{\color{gray}\small (reducible forms) }
				& \text{if } n-1\leq k < 0 \\
				(\Omega(M)_{ham}^{n-1})_{[N]} 
				\quad
				\text{\color{gray}\small (reducible hamiltonians) }
				& \text{if } k = 0 \\
				0 & \text{if } k > 0
			\end{cases}
		\end{displaymath}
	\end{defpropblock}

	\begin{defpropblock}[Vanishing $L_\infty$-observables]
		Is the {\color{blue!70!black}$L_\infty$-ideal} of $L_\infty(M,\omega)_{[N]}$ given by
		\begin{displaymath}
			I_{L_\infty(M,\omega)} :=
			\left\lbrace
				\alpha \in L_\infty(M,\omega)_{[N]}
			~\left\vert~
				\begin{array}{l l}
					\alpha(v_1,\dots,v_k) \in I_N  \quad \forall v_i \in \X_N &
					\text{if}~ \alpha \in \Omega^k \\
					\vHam_\alpha \in \X_\g + I_\X &
					\text{if}~ \alpha \in \Omega^{n-1}
				\end{array}
			\right\rbrace\right.
		\end{displaymath}
	\end{defpropblock}

	\begin{defblock}[Reduced $L_\infty$-algebra of observables]
		Is the $L_\infty$-quotient : \quad
		$\dfrac{L_\infty(M,\omega)_{[N]}^k}{I_{L_\infty(M,\omega)}}$
	\end{defblock}

\end{frame}
\note[itemize]{
 \item
}
%-------------------------------------------------------------------------------------------------------------------------------------------------

%-------------------------------------------------------------------------------------------------------------------------------------------------
\begin{frame}{Singular multisymplectic reduction}

\end{frame}
\note[itemize]{
 \item
}
%-------------------------------------------------------------------------------------------------------------------------------------------------



%-------------------------------------------------------------------------------------------------------------------------------------------------
\subsection{Homotopy comomentum maps}\label{frame:hcmm-main}
\begin{frame}[fragile]{Homotopy comomentum maps}
	Consider a Lie algebra action $v:\mathfrak{g} \to \mathfrak{X}(M)$  \underline{preserving the $n$-plectic form $\omega$}.
	\vfill
	\begin{defblock}[Homotopy comomentum map \emph{(Callies, Fregier, Rogers, Zambon)}]
		\ifHandout
			\includestandalone{Pictures/Figure_Lifting}
		\else
			\includestandalone{Pictures/Frame_Lifting}
		\fi					
	\end{defblock}
	\onslide<4->{
	\begin{lemblock}[HCMM unfolded  (CFRZ16)]
			%
			HCMM is a sequence of (graded-skew) multilinear maps:
			\begin{displaymath}
				(f)  = \big\lbrace f_k: \; \Lambda^k{\mathfrak g} \to L^{1-k} \subseteq \Omega^{n-k}(M) 
				~\big\vert~ 0\leq k \leq n+1  \big\rbrace
			\end{displaymath}
			\emph{fulfilling:}%\emph{such that:}
			\begin{itemize}
				\item<5-> $f_0 = 0 $, $f_{n+1} = 0$
				\item<6-> $d f_k (p) = f_{k-1} (
				\tikz[baseline,remember picture]{\node[rounded corners,
                        fill=green!5,draw=green!30,anchor=base]            
            			(target) {$\partial $ };
            	}				
				p)  - (-1)^{\frac{k(k+1)}{2}} \iota(v_p) \omega 
				\qquad\scriptstyle \forall p \in \Lambda^k(\mathfrak{g}),\; \forall k=1,\dots n+1$
			\end{itemize}
		\onslide<7->{
			\tikz[overlay,remember picture]
			{
				\node[rounded corners,
	                 draw=green!30,anchor=base]            
	            	 (base) at ($(current page.east)-(3,3)$) [rotate=-0,align=center] {\footnotesize{\hyperlink{frame:CE-complex}{\emph{Chevalley-Eilenberg boundary op.}}}};
			}	
		\begin{tikzpicture}[overlay,remember picture]
	    	\path[->] (base.west) edge[bend right,green](target.north east);
	    \end{tikzpicture}
	    }
	\end{lemblock}	
	}
	\vfill
\end{frame}
\note[itemize]{
	\item  An infinitesimal symmetry is a lie algebra morphism such that $\mathcal{L}_{v_x} \omega = 0 ~ \forall x \in \mathfrak{g}$.
	\\ (It is also call an infinitesimal multisymplectic action and $v_x$ is the infinitesimal generator of the action, corresponding to $x \in \mathfrak g$.) 
	\item Essentially, admitting a comoment maps mean that $v$ acts via Hamiltonian vector fields.
	\item In mechanical terms, a moment map is a tool associated with a Hamiltonian action of a Lie group on a symplectic manifold, used to construct conserved quantities for the action.(see \ref{frame:HCMMandConserved} in appendix.
}
%-------------------------------------------------------------------------------------------------------------------------------------------------

%-------------------------------------------------------------------------------------------------------------------------------------------------
\begin{frame}[fragile,shrink]{Homotopy co-moment maps \emph{(Callies, Fregier, Rogers, Zambon)}}
	\begin{columns}
		\begin{column}{.625\linewidth}	
			HCMM is an $L_\infty$-morphism  $\quad(f):\mathfrak{g}\to L_\infty(M,\omega)$
			\\[.5em]
			 lifting the  infinitesimal action $\quad v:\mathfrak{g}\to \mathfrak{X}(M)$
		\end{column}
		\begin{column}{.325\linewidth}	
			\begin{displaymath}
				\begin{tikzcd}[column sep = large]
					& L_{\infty}(M,\omega) \ar[d,"\mathscr{v}"]
					\\
					\mathfrak{g} \ar[ur,dashed,"(f)"]\ar[r,"v"']& \mathfrak{X}(M)
				\end{tikzcd}	
			\end{displaymath}		
		\end{column}
	\end{columns}
	\pause
	\begin{lemblock}[HCMM unfolded  \cite{Callies2016}]
			%
			HCMM is a sequence of (graded-skew) multilinear maps:
			\begin{displaymath}
				(f)  = \big\lbrace f_k: \; \Lambda^k{\mathfrak g} \to L_{k-1} \subseteq \Omega^{n-k} 
				\;\big\vert\; 0\leq k \leq n+1  \big\rbrace
			\end{displaymath}
			%
			\vspace{-.5em}	
			\includestandalone[width=0.9\textwidth]{Pictures/Frame_HCMM}
			
			\vspace{-1em}		
			\emph{fulfilling:}%\emph{such that:}
			\begin{itemize}
				\item<2-> $f_0 = 0 $, $f_{n+1} = 0$
				\item<3-> $d f_k (p) = f_{k-1} (\partial p)  - (-1)^{\frac{k(k+1)}{2}} \iota(v_p) \omega 
				\qquad\scriptstyle \forall p \in \Lambda^k(\mathfrak{g}),\; \forall k=1,\dots n+1$
			\end{itemize}
		\end{lemblock}

	\begin{tamblock}
	 Practically a HCMM is given by several multilinear maps
	 \begin{displaymath}
	 	f_i = \Lambda^i \mathfrak{g} \to L_{i-1}
	 \end{displaymath}
	 satisfying:
	 \begin{enumerate}
	 	\item $ d f_1(\xi) = - \iota_{v_\xi} \omega$
	 	\item $\sum ...$
	 \end{enumerate}
	\end{tamblock}


\end{frame}
\note[itemize]{
	%\item 		Consider:  $v:\mathfrak g\to \mathfrak X(M)$  a Lie algebra morphism  s.t. $\mathcal{L}_{v_x}\omega=0 \quad  \forall x\in\mathfrak g$ (i.e infinitesimal multisymplectic Lie algebra action $\mathfrak{g}\circlearrowleft (M,\omega)$)
	\item More conceptually, a comoment is an $L_\infty$-morphism $(f):\mathfrak{g}\to L_\infty(M,\omega)$ lifting the action $v:\mathfrak{g}\to \mathfrak{X}(M)$, 
i.e. making the diagram commute in the $L_\infty$-algebras category.
	\item The vertical arrow is the trivial $L_\infty$-extension of the function mapping any Hamiltonian form to the unique corresponding Hamiltonian vector field (an it is zero elsewhere)
		\\
		(Note that any Lie algebra can be seen as an $L_\infty$-algebra concentrated in degree $0$, therefore any $L_\infty$-morphism $L\to\mathfrak{g}$ is simply given by a linear map $L_0 \to \mathfrak{g}$ preserving the binary brackets.)
	\item We will make use of an explicit version of this definition which is expressed by the lemma.
	 Practically speaking, a HCMM is given by several multilinear maps ...
	 \item In the equation we have tacitly set $\Lambda^{-1}(M) = 0$
	 %\item Notation: \qquad $\partial =$ Chevalley-Eilenberg boundary operator.
	%\item Notice that a HCMM pertains to an "infinitesimal" action of ${\mathfrak g}$ on $M$ with ${\mathfrak g}$ being the Lie algebra of a generic Lie group $G$, acting on $M$ by $\omega$-preserving vector fields.
		\item (Notation) $ p = \xi_1 \wedge \xi_2 \wedge \dots \wedge \xi_k$, 
			then $v_p = v_1 \wedge v_2 \wedge \dots \wedge v_k$ 
			where $v_i \equiv v_{\xi_i}$ are the fundamental vector fields associated to the action $G \circlearrowright M$.
	%	\item (Notation) $\iota(v_p) \omega = \iota(v_k)\dots\iota(v_1) \omega$
	%	\item $\varsigma(k) := - (-1)^{\frac{k(k+1)}{2}}$ 
		\item (Notation) $(\iota^{k}_{\mathfrak{g}}\omega)(p):= \iota(v_p) \omega = \iota(v_k)\dots\iota(v_1) \omega$
		\item $\partial \equiv \partial_k:  \Lambda^{k} {\mathfrak g} \to \Lambda^{k-1} {\mathfrak g}$  is the usual Eilenberg-Chevalley complex boundary operator (see appendix, pag: \ref{frame:CE-complex});
%		\item The definition tells us that the {\it closed} forms
%			$$\mu_k := f_{k-1} (\partial p) +  \varsigma(k) \iota(v_p) \omega 	$$
%			must actually be {\it exact}, with potential $-f_k(p)$.  	
		\item The last equation tells us that an HCMM is not a chain complex morphism but is rather a chain complex homotopy between 0 and the multicontraction $\alpha=(\iota^{k}_{\mathfrak{g}}\omega)$ (see next slide).
		is a chain map by lemma 2.18 \cite{Ryvkin2016}).
}
%---------------------------------------------------------------------------------------------------------------------------------

%-------------------------------------------------------------------------------------------------------------------------------------------------
\begin{frame}[t]{Symmetries in \textbf{multisymplectic geometry}}
	Consider a Lie algebra action $v:\mathfrak{g} \to \mathfrak{X}(M)$  preserving the $n$-plectic form $\omega$,
	\vfill

	\vspace{-1em}
	\begin{columns}[T]
		\setlength{\belowdisplayskip}{5pt}
		\begin{column}{.50\linewidth}
			%
			\centering \it
			\onslide<2->{
				$-$ symplectic case $-$
				\begin{defblock}[Comoment map pertaining to $v$]
					Lie algebra morphism
					$$ f: \mathfrak{g} \to C^\infty(M) $$
					such that
					$$ d~f (x) = -\iota_{v_x} \omega \qquad \forall x \in \mathfrak{g}~.$$
				\end{defblock}
			}
		\end{column}	
		%
		\onslide<2->{\vrule{}}
		%
		\begin{column}{.50\linewidth}
			\centering \it
			\onslide<3->{			
				$-$ $n$-plectic case $-$
				\begin{defblock}[Homotopy comoment map \tiny (HCMM)]
					$L_\infty$-morphism 
					$$ (f_k) : \mathfrak{g} \to L_\infty (M,\omega)$$
					such that
					$$ d~f_1(x) = -\iota_{v_x} \omega \qquad \forall x \in \mathfrak{g}~.$$
				\end{defblock}	
			}
		\end{column}	
	\end{columns}	
	%
	\pause
	\vfill
	\centering 
	\onslide<4->{\textbf{-- Conserved quantities --}}
	%
	\vspace{-.5em}
	\begin{columns}[T]
		\setlength{\belowdisplayskip}{5pt}
		\begin{column}{.50\linewidth}
			%
			\centering \it
			\onslide<4->{
			\begin{propblock}[Noether Theorem]
				\small Fixed $H\in C^\infty_{\text{Ham}}(M)$ ($\mathfrak{g}$-invariant) ,
				$$\mathcal{L}_{v_H} f(x) = 0 \qquad \forall x \in \mathfrak{g}$$
			\end{propblock}
			}
		\end{column}	
		%
		\onslide<5->{\vrule{}}
		%
		\begin{column}{.50\linewidth}
			\centering \it
			\onslide<5->{			
			\begin{propblock}[RWZ16 Theorem]
				\small Fixed $H\in \Omega^{n-1}_{\text{Ham}}(M)$ ($\mathfrak{g}$-invariant),
				$$\mathcal{L}_{v_H} f_k(p) \in B^k(M) \qquad \forall p \in Z_k(\mathfrak{g})$$			
			\end{propblock}
			}
		\end{column}	
	\end{columns}		
\end{frame}
\note{
}
%-------------------------------------------------------------------------------------------------------------------------------------------------



\end{document}