%- HandOut Flag -----------------------------------------------------------------------------------------
\makeatletter
\@ifundefined{ifHandout}{%
  \expandafter\newif\csname ifHandout\endcsname
}{}
\makeatother

%- D0cum3nt ----------------------------------------------------------------------------------------------
\documentclass[beamer,10pt]{standalone}   
%\documentclass[beamer,10pt,handout]{standalone}  \Handouttrue  

\ifHandout
	\setbeameroption{show notes} %print notes   
\fi

	
%- Packages ----------------------------------------------------------------------------------------------
\usepackage{custom-style}
\usetikzlibrary{positioning}
\usepackage{multicol}


%--Beamer Style-----------------------------------------------------------------------------------------------
\usetheme{toninus}
\usepackage{animate}
\usetikzlibrary{positioning, arrows}
\usetikzlibrary{shapes}


% ==========================================================================================
\begin{document}
% ==========================================================================================

%-------------------------------------------------------------------------------------------------------------------------------------------------
\begin{frame}[fragile]{Keywords}
\tikzstyle{every picture}+=[remember picture]
	\begin{columns}
    	\begin{column}{.45\textwidth}
    		\onslide<3->{
			\tikz[baseline]{
		            \node[draw=orange!40,anchor=base,text width=5cm] (s1)
		            {Group of transformations preserving the prescribed geometric structures.};
			}
		}
	\end{column}
    	\begin{column}{.45\textwidth}
    		\onslide<4->{
			 \tikz[baseline]{
		            \node[draw=blue!40,anchor=base, text width=5cm] (s2)
		            {Procedure providing another manifold of reduced dimension encoding the relevant geometrical information.};
			}
		}
	\end{column}
	\end{columns}

	\vfill

	\begin{center}
		\large
		 \tikz[baseline]{
		            \node[fill=orange!20,anchor=base] (t1)
		            {Symmetries};
			}
		 and
		 \tikz[baseline]{
		            \node[fill=blue!20,anchor=base] (t2)
		            {Reduction};
		        } 
		\\
		of 
		 \tikz[baseline]{
	            \node[fill=green!20,anchor=base] (t3)
	            {Multi};
		}
		\kern-3pt - \kern-3pt
		 \tikz[baseline]{
	            \node[fill=red!20,anchor=base] (t4)
	            {Symplectic};
		}		
        manifolds
	\end{center}

	\vfill

	\begin{columns}
    	\begin{column}{.45\textwidth}
    		\onslide<2->{
	 		 \tikz[baseline]{
	            \node[draw=green!40,anchor=base,text width=5cm] (s3)
	            {A certain higher version \\(involving differential forms in degree $\geq 2$)};
	         }
		}		   	
		\end{column}
    	\begin{column}{.45\textwidth}
		\tikz[baseline]{
	            \node[draw=red!40,anchor=base,text width=5cm] (s4)
	            {Geometric structure providing a prescription on how to measure the area of 2-dimensional surface elements embedded in the manifold.};
		}	
		\end{column}
	\end{columns}

	\begin{tikzpicture}[overlay]
        \path[->,draw=orange!40]<3-> (s1) edge [bend right] (t1);
        \path[->,draw=blue!40]<4-> (s2) edge [bend left] (t2);
        \path[->,draw=green!40]<2-> (s3) edge [bend left] (t3);
        \path[->,draw=red!40] (s4) edge [bend right] (t4);
	\end{tikzpicture}

\end{frame}
\note[itemize]{
	\item Provide a vague idea of the words that make up the title.
}
%-------------------------------------------------------------------------------------------------------------------------------------------------


%-------------------------------------------------------------------------------------------------------------------------------------------------
\begin{frame}{Symplectic geometry (mechanics flavour)}
\begin{columns}[T]
	\begin{column}{.50\linewidth}
		\centering
		\textit{ "geometric approach" to mechanics \dots}
		%
		\begin{columns}
			\begin{column}{.60\linewidth}
				\begin{center}
					\includegraphics[width=0.6\linewidth]{Pictures/pendulum13}			
				\end{center}
			\end{column}	
			\begin{column}{.40\linewidth}
				\begin{center}
					\includegraphics[width=0.45\linewidth]{Pictures/pendulum-phase-space}			
				\end{center}
			\end{column}	
		\end{columns}
		%
		\begin{defblock}[Symplectic Manifold]
			\vspace{-1em}
			\includestandalone[width=1\textwidth]{Pictures/Figure_sym}	
		\end{defblock}
		%
		\pause
		\begin{exblock}[$M = T^\ast Q$ is symplectic]
			with $\omega = d \theta $ given by
			$$ \left.\theta\right\vert_{(q,p)} (v) = p (\pi_\ast v) ~.$$
		\end{exblock}
		%
		\pause
		\vspace{1em}
		\centering
		\textit{ based on the notion of \\"states".}		
	\end{column}
	\onslide<1->{\vrule{}}
	\pause
	\begin{column}{.50\linewidth}
		\centering
		\textit{ "algebraic approach" to mechanics \dots}
		\vspace{.5em}	
		\begin{defblock}[Classical Observables]
			Unital, associative, commutative algebra $C^\infty(M)$.
		\end{defblock}
		%
		\vspace{.5em}
		\pause
		\begin{defblock}[Hamiltonian vector fields]
			$v_f \in \mathfrak{X}(M)$ such that:
			$$\iota_{v_f} \omega = -df \quad \text{(exact)}$$ %$\in B^1(M)$
			\small$v_f$ = \emph{Ham.v.f. pertaining to $f\in C^\infty(M)$}.
		\end{defblock}
		%
		\begin{defblock}[Poisson Algebra of Observables]
			$C^\infty(M)$ is a Poisson algebra with
			$$\{f,g\} = \iota_{v_g} \iota_{v_f} \omega = \omega(v_f,v_g) ~.$$
		\end{defblock}
		%
		\pause
		\vspace{.8em}
		\centering
		\textit{ based on the notion of \\"meaurable quantities".}						
	\end{column}
\end{columns}
\end{frame}
\note[itemize]{
	\footnotesize

	\item We work in the framework of multisymplectic geometry which is one of the possible generalizations of the well-established field of symplectic geometry.
	
	\item To recall what symplectic geometry is let me assume a particular point of view: mechanics.
	\\
	Idea:"
	Symplectic geometry is a branch of differential geometry studying symplectic manifolds; it originated as a formalization of the mathematical apparatus of classical mechanics and geometric optics."{\href{https://ncatlab.org/nlab/show/symplectic+geometry}{nlab}}
	
	Namely, a sym. mfd. is the geometric structure encoding the phase space of conservative, ordinary, classical, mechanical systems.
	
	\item $\theta$ = \emph{tautological 1-form}.
		$\theta$ evaluated at $p\in T^*Q$ in the fibre of $q\in Q$ and contracted with $v$ coincides with the form $p$ evaluated at $q$ and contracted with the push forward of $v$.
	
	\item We identify a special class of vector fields.
		Out of them one can define a Lie bracket.
	
	\item Poisson is a Lie algebra with the extra property of compatibility with the associative product (Leibniz rule)
	
	\item take away message: geometric (based on "states") vs algebraic (based on "measurable quantities").7
}
%-------------------------------------------------------------------------------------------------------------------------------------------------



%-------------------------------------------------------------------------------------------------------------------------------------------------
\begin{frame}{From symplectic to {multi}symplectic} 
	%
	\begin{center}
		$-$ \emph{multisymplectic means \textbf{going higher} in the degree of $\omega$} $-$
	\end{center}
	\pause
	\begin{defblock}[$n$-plectic manifold ~\emph{(Cantrijn, Ibort, De Le\'on)} \cite{Cantrun2017}]
		\includestandalone[width=0.95\textwidth]{Pictures/Figure_multisym}	
	\end{defblock}
	%
	\pause
		\begin{table}
			\begin{tabular}{c c c}
				symplectic forms \small($n=1$) & $\leftrightsquigarrow$ & volume forms \small($n= \text{dim}(M)-1$)
			\end{tabular}
		\end{table}

	\vfill
	\pause
	\begin{block}{Historical motivation}
		Mechanics: geometrical foundations of \textit{(first-order)} field theories.
	\end{block}
	\begin{table}
		\only<4>{
		\begin{tabular}{|p{0.2\textwidth}|p{0.3\textwidth}|p{0.35\textwidth}|} 
            \hline
            \parbox[][20pt][c]{0.2\textwidth}{mechanics} & \multicolumn{2}{c|}{geometry} \\
            \hline
            \parbox[][20pt][c]{0.2\textwidth}{phase space} & symplectic manifold &  \\[.25em]
            \parbox[][20pt][c]{0.2\textwidth}{classical \\ observables} & Poisson algebra &  \\[.25em]
            \parbox[][20pt][c]{0.2\textwidth}{symmetries} &  group actions admitting comoment map &  
            \\
            \hline
  \multicolumn{1}{c}{}
            &  \multicolumn{1}{@{}c@{}}{$\underbrace{\hspace*{.3\textwidth}}_{\text{point-like particles systems}}$} 
            &  \multicolumn{1}{@{}c@{}}{}              \\
		\end{tabular}
		
		
		}
		\onslide<5->{
		\begin{tabular}{|p{0.2\textwidth}|p{0.3\textwidth}|p{0.35\textwidth}|} 
            \hline
            \parbox[][20pt][c]{0.2\textwidth}{mechanics} & \multicolumn{2}{c|}{geometry} \\
            \hline
            \parbox[][20pt][c]{0.2\textwidth}{phase space} & symplectic manifold & multisymplectic manifold \\[.25em]
            \parbox[][20pt][c]{0.2\textwidth}{classical \\ observables} & Poisson algebra & $L_\infty$-algebra \\[.25em]
            \parbox[][20pt][c]{0.2\textwidth}{symmetries} &  group actions admitting comoment map & group actions admitting (homotopy) comomentum map
            \\
            \hline
  			\multicolumn{1}{c}{}
            &  \multicolumn{1}{@{}c@{}}{$\underbrace{\hspace*{.3\textwidth}}_{\text{point-like particles systems}}$} 
            &
            \multicolumn{1}{@{}c@{}}{$\underbrace{\hspace*{.3\textwidth}}_{\text{field-theoretic systems}}$} 
               \\
		\end{tabular}
		}
	\end{table}	




\end{frame}
\note[itemize]{
	\item Historically, the interest in multisymplectic manifolds, has been motivated by the need for understanding the geometrical foundations of first-order classical field theories.
	The key point is that, just as one can associate a symplectic manifold to an ordinary classical mechanical system (e.g. a single
point-like particle constrained to some manifold), it is possible to associate a multisymplectic
manifold to any classical field system (e.g. a continuous medium like a filament or a fluid). See frame Extra-\ref{Frame:Ms-Field-Mechanics} 
	
	\item General ideas basic parallelisms with caveats
	\item caveat: points in multiphase spaces are not states
	\item the table hides the duality between geometric and algebraic approaches to the problem.
	\item	Mechanics: geometrical foundations of \textit{(first-order)} field theories.
		\begin{itemize}
		 \item[-] Kijowski, W. Tulczyjew \cite{Kijowski1979}; %(1979)
		 \item[-] Cariñena, Crampin, Ibort \cite{Carinena1991b};% (1991)
		 \item[-] Gotay, Isenberg, Marsden, Montgomery \cite{Gimmsy1};%(1998)
		 \\ $\cdots$
		\end{itemize}
}
%-------------------------------------------------------------------------------------------------------------------------------------------------






% ==========================================================================================
\end{document}
% ==========================================================================================
