%- HandOut Flag -----------------------------------------------------------------------------------------
\makeatletter
\@ifundefined{ifHandout}{%
  \expandafter\newif\csname ifHandout\endcsname
}{}
\makeatother

%- D0cum3nt ----------------------------------------------------------------------------------------------
\documentclass[beamer,10pt]{standalone}   
%\documentclass[beamer,10pt,handout]{standalone}  \Handouttrue  

\ifHandout
	\setbeameroption{show notes} %print notes   
\fi

	
%- Packages ----------------------------------------------------------------------------------------------
\usepackage{custom-style}
\usepackage{math}
\usetikzlibrary{positioning}
\usepackage{multicol}
\usepackage{bbold}

%--Beamer Style-----------------------------------------------------------------------------------------------
\usetheme{toninus}
\usepackage{animate}
\usetikzlibrary{positioning, arrows}
\usetikzlibrary{shapes}


%============================================================================================================
\begin{document}
%============================================================================================================


%-------------------------------------------------------------------------------------------------------------------------------------------------
\subsection{Multisymplectic manifolds}
%-------------------------------------------------------------------------------------------------------------------------------------------------
\begin{frame}[fragile]{Multisymplectic manifolds} %Fragile -->workaround tikzcd
	\begin{defblock}[$n$-plectic manifold ~\emph{(Cantrijn, Ibort, De Le\'on)}]
		\includestandalone[width=0.95\textwidth]{Pictures/Figure_multisym}	
	\end{defblock}
	%
	\begin{defblock}[Non-degenerate $(n+1)$-form]
		\begin{columns}
			\begin{column}{.45\linewidth}
				\centering{
				The $\omega^\flat$ (flat) bundle map is injective.
				}
			\end{column}
			\begin{column}{.5\linewidth}
						\vspace{-.5em}
				\[
				\begin{tikzcd}[column sep= small,row sep=0ex,
				/tikz/column 1/.append style={anchor=base east}]
				    \omega^\flat \colon T M \ar[r]& \Lambda^n T^\ast M \\
  						 (x,u) \ar[r, mapsto]& (x,\iota_{u} \omega_x)						
				\end{tikzcd}	
				\]
			\end{column}
		\end{columns}
	\end{defblock}
	%
	\vfill
	%
	\begin{block}{Examples:}
		\vspace{-.5em}
		\setbeamercovered{transparent}
		\begin{itemize}[<+->]
			\item[$\bullet$] $n=1$ \qquad\qquad\qquad $\Rightarrow$\quad $\omega$ is a symplectic form
			\item[$\bullet$]  $n=(dim(M)-1)$ \quad$\Rightarrow$\quad $\omega$ is a volume form
			%Any oriented $(n+1)$-dimensional manifold is $n$-plectic w.r.t. the volume form.
			%\item[$\bullet$] Let $G$ a semisimple Lie group and $\langle\cdot,\cdot \rangle$  its killing form. Then $\langle [\cdot,\cdot],\cdot \rangle$ extends to a biinvariant multisymplectic form $\omega$.
			\item[$\bullet$] Let $Q$ a smooth manifold, the multicotangent bundle $\Lambda^n T^\ast Q$ is naturally $n$-plectic.%
			\quad
			\textit{(cfr, \href{https://arxiv.org/abs/physics/9801019}{GIMMSY} construction for classical field theories)}
		\end{itemize}
	\end{block}			 
	
%
\end{frame}
\note[itemize]{
	\item Multisymplectic ($n$-plectic) geometry is a generalization of symplectic geometry where a closed, non degenerate $n+1$-form $(n\geq 1)$  takes the place of the symplectic 2-form
	
	\item multisymplectic means \emph{going higher} in the degree of $\omega$
	
	\item non degeneracy means $\iota_v\omega = 0 \Leftrightarrow v=0$.
	
	\item examples: 
	\\1-plectic $=$ symplectic, 
	\\any oriented $(n+1)$-dimensional manifold is $n$-plectic w.r.t. the volume form
	\\the multicotangent bundle $\Lambda^n T^\ast Q$ is naturally $n$-plectic (see frame extra-\ref{Frame:Ms-Field-Mechanics})
	
	\item Observe also that, by degree reason, when $n$ is equal to $1$ or $dim(M)+1$ an injective flat map $\flat$ is also bijective.
	
	\item It is important to stress that mechanical systems are not the only source instances of this class of of structures. 
				e.g. any semisimple Lie groups has associated a 2-plectic structure and any oriented $n+1$ dimensional manifold is naturally $n$-plectic.
				

}
%---------------------------------------------------------------------------------------------------------------------------------------------------



%------------------------------------------------------------------------------------------------
\begin{frame}[t]{Special classes of smooth objects}\label{Frame:Classes-Multisymplectic-Objects}
	\vspace{-1.5em}
  	\begin{columns}
		\begin{column}[t]{.42\linewidth}		
			\begin{defblock}[Hamiltonian v.f.]
				$\mathfrak{X}_{ham} =  \left\lbrace X \in  \mathfrak{X} \right\vert \left. \iota_x \omega \textrm{ exact}  \right\rbrace$ 			
			\end{defblock}
			\begin{defblock}[Multisymplectic v.f.]
				$\mathfrak{X}_{ms} =  \left\lbrace X \in  \mathfrak{X} \right\vert \left. \mathcal{L}_X \omega = 0  \right\rbrace$ 	
			\end{defblock}
		\end{column}
		\begin{column}[t]{.58\linewidth}		
			\begin{defblock}[Hamiltonian $(n$-$1)-$forms]
				\begin{displaymath}
					\Omega^{n-1}_{ham} 	:=
					\biggr\{ H \in  \Omega^{n-1} \; \left\vert \; 
					\stackanchor{$\exists X \in \mathfrak{X}_{ham}$}{: $d H = -\iota_X \omega$} \right\} 
			\end{displaymath}
			\end{defblock}		
		\end{column}
  	\end{columns}
  	%
  	\vfill
  	%
  	\onslide<2->{
  	\begin{columns}
		\begin{column}[t]{.5\linewidth}	
			\centering\emph{Global symmetries}
			\begin{defblock}[Multisym. (Lie group) action]
				Smooth action $\Phi: G \curvearrowright (M, \omega)$ s.t. 
				$$(\Phi_g)_\ast \omega = \omega \qquad \forall g \in G~.$$
			\end{defblock}
		\end{column}
		\begin{column}[t]{.5\linewidth}			
			\centering\emph{Infinitesimal symmetries}
			\begin{defblock}[Multisym. (Lie algebra) action]
				Lie algebra hom.
				$\vAct: \mathfrak{g} \rightarrow \mathfrak{X} (M)$  s.t.
				$$\mathcal{L}_{\vAct_\xi} \omega = 0 \qquad \forall \xi \in \mathfrak{g}~.$$	
			\end{defblock}
		\end{column}
  	\end{columns}
  	}
  	%
  	\vfill
  	\onslide<3->{		
	  	\begin{asideblock}[Hierarchy of conserved quantities]%Shades of...
	  		\begin{table}[] % http://tablesgenerator.com/
			\begin{tabular}{lllll}
					& strictly conserved & & & $\mathcal{L}_X \alpha= 0$ \\
				$\alpha \in \Omega^\bullet$ & globally conserved & along $X \in \mathfrak{X}$ & $\Leftrightarrow$ & $\mathcal{L}_X \alpha\in B $ (exact) \\
				  & locally conserved  & & & $\mathcal{L}_X \alpha\in Z $ (closed)                                
			\end{tabular}
			\end{table}
	  	\end{asideblock}
  	}
  	
  \end{frame}
  \note[itemize]{
  	\item Exactly as it happens in symplectic geometry, fixing a smooth form $\omega$ on $M$ yields a criterion for classifying vector fields and differential forms.
  	\\(Pay attention to the sign convention in defining the Hamiltonian vector fields)
	\item We recognize the special class of forms, called Hamiltonian, determining the Hamiltonian vector fields. 
	Not every $n-1$ form admits an Hamiltonian vector field.
	When it exists, non degeneracy guarantees unicity.  	
  	\item Also, we can naturally select a special class of symmetries (global and infinitesimal) which preserve the fixed multisymplectic form.
  	\item Aside, we can start to see that, in this setting, measurable quantities are not only smooth functions but also differential forms with degree greater then zero.
  	For such objects can be defined weaker notions of conservation along a flow.
  	\item The idea to consider forms of various degree as observables do not fall out of the sky. 
  		For instance in a string there will be two kind of measurable quantities: extensive observable (1-forms), like the density, and intensive observables (0-forms), like the tension. 
 		%\href{https://en.wikipedia.org/wiki/Intensive_and_extensive_properties#Intensive_properties}{(wiki link on this terminology)}
  	\item Starting from this observation we can define the space of all possible observables (see next slide).
  }
%---------------------------------------------------------------------------------------------------------------------------------------------------












%-------------------------------------------------------------------------------------------------------------------------------------------------
\subsection{Observability}
%-------------------------------------------------------------------------------------------------------------------------------------------------

%-------------------------------------------------------------------------------------------------------------------------------------------------
\begin{frame}{Observables in $n$-plectic geometry}
	%
	\begin{defblock}[Hamiltonian $(n-1)$-forms]
		\begin{displaymath}
			\Omega^{n-1}_{ham}(M,\omega) 	:=
			\biggr\{ \sigma \in  \Omega^{n-1}(M) \; \biggr\vert \; 
				\exists \vHam_\sigma \in \mathfrak{X}(M) ~:~ 
				\tikz[baseline,remember picture]{\node[rounded corners,
                        fill=orange!5,draw=orange!30,anchor=base]            
            			(target) {$d \sigma = -\iota_{\vHam_\sigma} \omega$ };
            	}				
				~\biggr\} 
			\end{displaymath}
	\end{defblock}

	\pause
		\tikz[overlay,remember picture]
		{
			\node[rounded corners,
                 fill=orange!5,draw=orange!30,anchor=base]
            	 (base) at ($(current page.north east)-(2,1)$) [rotate=-0,text width=3.5cm,align=center] {\footnotesize{\textcolor{red}{Hamilton-DeDonder-Weyl \\equation}}};
		}	
	\begin{tikzpicture}[overlay,remember picture]
    	\path[->] (base.south) edge[bend right,red](target.north);
    \end{tikzpicture}
	%
	\vspace{-1em}
	\begin{columns}[T]
		\pause
		\setlength{\belowdisplayskip}{5pt}
		\begin{column}{.50\linewidth}
			%
			\begin{thmblock}[Observables $L_\infty$-algebra]
				$\Omega^{n-1}_{ham}(M,\omega)$ endowed with
				\vspace{-.5em}
				\begin{displaymath}
					\lbrace \sigma_1, \sigma_2 \rbrace =			
					~ - \iota_{\vHam_1}\iota_{\vHam_2} \omega 
				\end{displaymath}			
				can be "completed" to a \\ $L_\infty-algebra$.
			\end{thmblock}
			%
			\pause
			\begin{itemize}
				\item[\cmark] Skew-symmetric;
				\item[\xmark] multiplication of observables;
				\item[\xmark] Jacobi equation;
				%\\ \hspace*{4.25em} full-fledged Jacobi equation;
				\item[\smark] Jacobi equation \emph{up to homotopies}.
			\end{itemize}				
		\end{column}	
		%
		\onslide<1->{\vrule{}}
		%
		\pause
		\begin{column}{.50\linewidth}
			\begin{thmblock}[Observables Leibniz algebra]
				$\Omega^{n-1}_{ham}(M,\omega)$ endowed with
				\vspace{-.5em}
			\begin{displaymath}
				\llbracket \sigma_1, \sigma_2 \rrbracket =
				~ \mathcal{L}_{\vHam_1} \sigma_2
				~.					
			\end{displaymath}		
				can be "completed" to a \\ DG-Leibniz algebra.
			\end{thmblock}	
			%
			\pause
			\begin{itemize}
				\item[\xmark] Skew-symmetric;
				\item[\xmark] multiplication of observables;
				\item[\cmark] Jacobi equation;
				\item[\smark] Skew-symmetric \emph{up to homotopies}.
			\end{itemize}	
		\end{column}	
	\end{columns}
\end{frame}
\note[itemize]{
 \item In \cite[Prop. 5.2]{Dehling17} it has been proved that $Leib(M,\omega)$ and $L_\infty(M,\omega)$ are equivalent as weak $L_\infty$-algebras at up to the 3-plectic case, the two-plectic one had already been proven in \cite[\S 7.2]{Rogers2010}.
 \item Notice that the failure of skew-symmetricity is measured by
 \begin{align*}
 	A(\sigma_1,\sigma_2) 
 	=&~
 	 2~\llbracket \sigma_1, \sigma_2\rrbracket +  \llbracket \sigma_2, \sigma_1\rrbracket
 	 \\
 	 =&~ d \circ\langle\sigma_1,\sigma_2\rangle_+
 \end{align*}
 (Compare with the \href{https://en.wikipedia.org/wiki/Courant_bracket\#Dorfman_bracket}{Dorfmann bracket}.)
 
 \item Notice that the antisymmetrization of $\llbracket \cdot, \cdot \rrbracket$ equates $\lbrace \cdot,\cdot \rbrace$ modulo homotopy:
 \begin{align*}
  \llbracket \sigma_1,\sigma_2 \rrbracket  \circ 	\mathcal{A}
  =&~
  \frac{\mathcal{L}_{\vHam_1}\sigma_2 - \mathcal{L}_{\vHam_2}\sigma_1}{2} 
  \\
  =&~ 
  \frac{1}{2}( d (\iota_{\vHam_1}{\sigma_2} - \iota_{\vHam_2}{\sigma_1} )
  +\frac{1}{2}(-\iota_{\vHam_1}\iota_{\vHam_2}\omega + \iota_{\vHam_2}\iota_{\vHam_1}\omega) 
  \\
  =&~
  \lbrace \sigma_1, \sigma_2 \rbrace + d \circ \langle\sigma_1,\sigma_2\rangle_-
  \\
  =&~
  [\sigma_1,\sigma_2]_\omega
 \end{align*}
 where the last bracket is given by the Higher Courant bracket.
 \item Take away: Two brackets on $\Omega_H^{k-1}(M)$, ${\color{red}\llbracket\alpha,\beta\rrbracket = \mathcal{L}_{X_\alpha} \beta}$ and ${\color{green}\{\alpha,\beta\}=\iota_{X_\alpha}\d\beta}$,
	Equal up to a coboundary: ${\color{red}\mathcal{L}_{X_\alpha}\beta} = {\color{green}\iota_{X_\alpha}\d\beta} + \d\hspace{1pt}\iota_{X_\alpha}\beta$
}
%-------------------------------------------------------------------------------------------------------------------------------------------------




%---------------------------------------------------------------------------------------------------------------------------------------------------
\subsection{$L_\infty$-algebra of Observables}
\begin{frame}[fragile,t]{$L_\infty$-algebra of Observables (higher observables) }
	Let be $(M,\omega)$ a $n$-plectic manifold.
	\begin{defblock}[$L_\infty$-algebra of observables ~\emph{(Rogers)}]
		\hspace{.25em} Is a cochain-complex $(L,\{\cdot\}_1)$ \\
		\vspace{-2.5em}
		\begin{center}
		\ifHandout
			\includestandalone{Pictures/Figure_Observables}	
		\else
			\includestandalone{Pictures/Frame_Observables}
		\fi				
		\end{center}
		\onslide<2->{
			\hspace{.25em} with $n$ (skew-symmetric) multibrackets $(2 \leq k \leq n+1)$\\
			\vspace{-1.5em}
			\begin{center}
				\includestandalone{Pictures/Equation_Multibracket}	
			\end{center}
		}
		%
	\end{defblock}
  \end{frame}
 \note[itemize]{
	\item if symplectic manifolds are the symmetric take on mechanics, Poisson algebras are the algebraic counterpart.
 	\item A Lie algebra is associated to an ordinary symplectic manifold (its Poisson algebra).
	%(Underlying this is a Lie algebra, whose Lie bracket is the Poisson bracket.)
	Similarly, one associates an Lie-$n$ algebra to any $n$-plectic manifold.
 	% https://ncatlab.org/nlab/show/n-plectic+geometry 	 
 	 %https://ncatlab.org/nlab/show/Poisson+bracket+Lie+n-algebra
	 \item Basically, the higher observables algebra is a chunk of the de Rham complex of $M$ with inverted grading( convention employed here) and an extra structure called "multibrackets".
 	\item ( In the 1-plectic case it reduces to the corresponding Poisson algebra of classical observables)
 	\item Rogers associated to any n-plectic mfd a $L\-\infty$ algebra, Zambon generalized it to the pre-n-plectic case.
 	\item Recognize in the definition of $\{\cdot,\ldots,\cdot\}_k$ the contraction with hamiltonian fields $v_\sigma$ w.r.t. $\sigma$.
  	\item Note $	\iota_{v_{\sigma_1}}\cdots\iota_{v_{\sigma_k}} = (-)^{(k-1)+(k-2)+\dots+1}\iota_{v_{\sigma_k}}\cdots\iota_{v_{\sigma_1}} = (-)^{\frac{k(k-1)}{2}}\iota_{v_{\sigma_k}}\cdots\iota_{v_{\sigma_1}}$ 
 	The definition usually find in literature of Rogers multibrackets involves the coefficient $ (-)^{\frac{k(k-1)}{2}} = -\varsigma(k-1) = (-)^{k+1} \varsigma(k)$.
  \item higher observables is Special instance of a more general object  called $L\-\infty$ Algebra...
 }
%------------------------------------------------------------------------------------------------

%------------------------------------------------------------------------------------------------
\begin{frame}{Reminder: $L_\infty$ Algebras}

		\emph{
			$L_\infty$-algebra is the notion that one obtains from a Lie algebra when one requires the Jacobi identity to be satisfied only up to a higher coherent chain homotopy.
		}
		\\
		\vspace{.5em}
		\begin{defblock}[$L_\infty$-algebra \emph{(Lada, Stasheff)}]
			\includestandalone{Pictures/Figure_Linfinitydef}
		\end{defblock}	
		%
		%
	\pause
	\vfill
	\begin{thmblock}[Rogers \cite{Rogers2010}]
		The \emph{higher observable algebra} $L_{\infty}(M,\omega)$ 	forms an honest $L_\infty$ algebra.
		\footnotetext{ Take $\mu_1 = \text{d}$, $L$ is a shifted truncation of the de Rham complex.}
	\end{thmblock}
\end{frame}
\note[itemize]{
	\item $L_\infty$-algebra is the notion obtained from a Lie algebra requiring that the Jacobi identity is satisfied only up to a higher coherent chain homotopy.
	\item The Lie-n algebra mentioned before is a $L_\infty$ algebra with underlying graded vector space concentrated in degrees $0,1...n$.
	
	\item Definition. We say that a permutation $\sigma \in S_n$ is a $(j,n-j)$-unshuffle, $0\leq j \le1 n$  if $\sigma(1)< \dots < \sigma(j)$ and $\sigma(j+1)<\dots<\sigma(n)$.
	\\
	You can also say that $\sigma$ is a $(j,n-j)$-unshuffle if $\sigma(i)< \sigma(i+1)$ when $i\neq j$.

	\item 	Alternatively, the Jacobiators can be also denoted as $$\displaystyle J_m=\sum_{i+j=m+1} 	\mu_i \circ \mu_j = 0$$
	employing the so-called \emph{ Richardson-Nijenhuis product}
		 $$\mu_i\circ \mu_j := (-)^{i(j+1)}\frac{1}{j!(i-1)!}\mu_i\circ\mu_j\otimes \mathbb{1}_{i-1} \circ \mathcal{A}~,$$
		 where $\mathcal{A}$ denotes the (graded) total skew-symmetrizator.
		 
	\item see frame extra-\ref{Frame:unwapping-Jacobi} for a slightly demystification of the higher Jacobi equations.

	\item more precisely this statement is a proposition/definition

}
%------------------------------------------------------------------------------------------------



%---------------------------------------------------------------------------------------------------------------------------------------------------
\subsection{Leibniz-algebra of Observables}
\begin{frame}[fragile,t]{Leibniz-algebra of Observables}


\end{frame}
\note[itemize]{
	\item in addition to being extendable to an $L_\infty$-algebra, the space $\Omega^{n-1}_{ham}(M)$ also carries the structure of a Leibniz algebra. This structure is very important for multisymplectic reduction procedures.
	\item 
}
%---------------------------------------------------------------------------------------------------------------------------------------------------

%============================================================================================================
\end{document}
%============================================================================================================