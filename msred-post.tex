%- HandOut Flag -----------------------------------------------------------------------------------------
\makeatletter
\@ifundefined{ifHandout}{%
  \expandafter\newif\csname ifHandout\endcsname
}{}
\makeatother

%- D0cum3nt ----------------------------------------------------------------------------------------------
\documentclass[beamer,10pt]{standalone}   
%\documentclass[beamer,10pt,handout]{standalone}  \Handouttrue  

\ifHandout
	\setbeameroption{show notes} %print notes   
\fi

	
%- Packages ----------------------------------------------------------------------------------------------
\usepackage{custom-style}
\usepackage{math}
\usetikzlibrary{positioning}
\usepackage{multicol}
\usepackage{xfrac}

%--Beamer Style-----------------------------------------------------------------------------------------------
\usetheme{toninus}
\usepackage{animate}
\usetikzlibrary{positioning, arrows}
\usetikzlibrary{shapes}

\renewcommand{\action}{\curvearrowright} 
\makeatletter
\def\blfootnote{\gdef\@thefnmark{}\@footnotetext}
\makeatother

\newcommand{\fgmodule}{\mathfrak{X}_{\g}}
\newcommand{\tgtvanform}{I_{\Omega}(N)}
\newcommand{\tgtvf}{\mathfrak X_N(M)}
\newcommand{\vanvf}{I_{\mathfrak{X}}(N)}%{I_N^{\mathfrak X}}
\newcommand{\Ham}{\mathsf{Ham}}
\renewcommand{\ham}{\text{ham}}

\begin{document}

Parti aggiunte all'ultimo per il talk a Brescia.


%-------------------------------------------------------------------------------------------------------------------------------------------------
\begin{frame}{Singular reduction} %BRESCIA VERSION
	Assume $\omega$ \emph{reducible} w.r.t. $G\action M$ and $N$, i.e.
	\begin{displaymath}
			\begin{array}{l l l r}
				\underline{\xi} \in \mathfrak{X}(M)_N ~,&
				\mathcal{L}_{\underline{\xi}} \omega \in I_\Omega ~,&
				\iota_{\underline{\xi}} \omega \in I_\Omega & 
				\tiny
				(\forall \xi \in \mathfrak{g})
			\end{array}
	\end{displaymath}

	%
	\pause
	%
	\begin{defpropblock}[Reduced Leibniz algebra of observables]
\[
	Leib(M,\omega)_N =
	\frac{
		\left\lbrace
			\alpha \in \Omega_{ham}^{n{-}1}(M)
		~\left\vert~
			\onslide<3->{
			\begin{array}{l l}
				\iota_{\underline \xi} \alpha \in I_{\Omega}(N) &
				\\
				\L_{\underline \xi} \alpha \in I_{\Omega}(N)
				&
				\\
				\L_{\underline \xi} v_\alpha \in \fgmodule +\vanvf
				&~\forall \xi \in \g 
				\\
				v_\alpha \in \X_N(M)
				&
			\end{array}
			}
		\right.
		\right\rbrace
	}{
		\left\lbrace
			\alpha \in \Omega_{ham}^{n{-}1}(M)
		~\left\vert~
			\begin{array}{l}
				\alpha \in I_{\Omega}(N)
				\\
				v_\alpha \in \fgmodule +\vanvf
			\end{array}
		\right.
		\right\rbrace
	}~.
\]

	\end{defpropblock}
	%
	\onslide<4->{
	%
	\begin{thmblock}[Singular red. is compatible with regular red.]
		Let be $G\action M$ hamiltonian, $\mu$ a moment map and $N=\mu^{-1}(\phi)$ a submanifold.
		\\
		Then:
		$$ Leib\left(\frac{N}{G_\phi},\omega_\phi\right) \cong Leib(M,\omega)_N~.$$
	\end{thmblock}
	}
\end{frame}
\note[itemize]{
	\item
}
%-------------------------------------------------------------------------------------------------------------------------------------------------



\end{document}

