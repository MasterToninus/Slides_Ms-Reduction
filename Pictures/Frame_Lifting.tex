%+------------------------------------------------------------------------+
%| Diagram: HCMM as a Lift
%| Author: Antonio miti
%+------------------------------------------------------------------------+


\documentclass[beamer]{standalone}
\usepackage{tikz-cd}
\usepackage{mathtools}
\usepackage{amsfonts}
\usetikzlibrary{decorations.pathmorphing}
	%Credit: https://tex.stackexchange.com/questions/99119/beamer-problematic-use-of-visible-and-only-in-combination-with-tikz-to-draw-a
  \tikzset{
    invisible/.style={opacity=0},
    visible on/.style={alt=#1{}{invisible}},
    alt/.code args={<#1>#2#3}{%
      \alt<#1>{\pgfkeysalso{#2}}{\pgfkeysalso{#3}} % \pgfkeysalso doesn't change the path
    },
  }
\RequirePackage[cal=boondox,scr=boondoxo]{mathalfa}

\begin{document}
	\begin{columns}
		\begin{column}{.325\linewidth}
				\begin{tikzcd}[column sep = large,row sep=small]
					& L_{\infty}(M,\omega) 
					\ar[dd,visible on=<2>,"\mathscr{v}"]
					\ar[d,visible on=<3->,two heads,"\pi_{Ham}"]
					\\
					& \only<2>{\phantom{\mathfrak{X}_{Ham}(M,\omega)}}
					\only<3->{
						\tikz[baseline,remember picture]{\node[rounded corners,
                        fill=blue!2,draw=blue!30,anchor=base]            
            			(target) {$\mathfrak{X}_{Ham}(M,\omega)$ };
            			}	
					} 
					\ar[d,hook,visible on=<3->]
					\\
					\mathfrak{g} \ar[uur,dashed,"(f)"]\ar[r,"v"']
					& \mathfrak{X}(M)
				\end{tikzcd}	
		\end{column}
		\begin{column}{.625\linewidth}	
			HCMM is an $L_\infty$-morphism  $~(f):\mathfrak{g}\to L_\infty(M,\omega)$
			\\[.8em]
			\onslide<2->{
			 lifting the  infinitesimal action $\quad v:\mathfrak{g}\to \mathfrak{X}(M)$}
			\\[1em]				 
			 \onslide<3->{ 
				\qquad\tikz[baseline,remember picture]{\node[rounded corners,
                        fill=blue!2,draw=blue!30,text width=.9\linewidth,align=center]            
            			(base) {
	            			\footnotesize{
            					\emph{(acting via Hamiltonian vector fields!)}
            					\\
            					$ d~f_1(x) = -\iota_{v_x} \omega \qquad \forall x \in \mathfrak{g}~.$
						}
            			};
            	}			
			 }

		\end{column}
	\end{columns}
	\only<3->{
	\begin{tikzpicture}[overlay,remember picture]
    	\path[->] (base.west) edge[bend left,blue](target.south east);
    \end{tikzpicture}		}
\end{document}

