%+----------------------------------------------------------------------------+
%| SLIDES: 
%| Chapter: Complementary material - details on eventual questions
%| Author: Antonio miti
%| Event: PHD preliminary Defence
%+----------------------------------------------------------------------------+

%- HandOut Flag -----------------------------------------------------------------------------------------
\newif\ifHandout

%- D0cum3nt ----------------------------------------------------------------------------------------------
\documentclass[beamer,10pt]{standalone}   
%\documentclass[beamer,10pt,handout]{standalone}  \Handouttrue  

%- HandOut Flag -----------------------------------------------------------------------------------------
\ifHandout
	\setbeameroption{show notes} %print notes   
\fi

	
%- Packages ----------------------------------------------------------------------------------------------
\usepackage{custom-style}

%--Beamer Style-----------------------------------------------------------------------------------------------
\usetheme{toninus}



\providecommand{\blank}{\text{\textvisiblespace}}


\newcommand{\subsectiontitle}{
  \begin{frame}
  \vfill
  \centering
  \begin{beamercolorbox}[sep=8pt,center,shadow=true,rounded=true]{title}
    \usebeamerfont{title}\insertsectionhead\par%
    \usebeamerfont{title}\insertsubsectionhead\par%
  \end{beamercolorbox}
  \vfill
  \end{frame}
}

\providecommand{\blank}{\text{\textvisiblespace}}




%---------------------------------------------------------------------------------------------------------------------------------------------------
%- D0cum3nt ----------------------------------------------------------------------------------------------------------------------------------
\begin{document}
%------------------------------------------------------------------------------------------------

%##################################################################################
\begin{frame}
	\begin{center}
	\Huge\emph{Supplementary Material}
	\end{center}
\end{frame}
\note[itemize]{
	\item
}
\addtocounter{framenumber}{-1}
%##################################################################################



%------------------------------------------------------------------------------------------------
\begin{frame}[fragile]{MS geometry and classical field mechanics}\label{Frame:Ms-Field-Mechanics}
		Consider a smooth manifold $Y$,
		\begin{columns}
			\hfill
			\begin{column}{.5\linewidth}
				\emph{Multicotangent bundle} $\bigwedge = \bigwedge^n T^\ast Y$\\
				is naturally $n$-plectic
			\end{column}
			\begin{column}{.4\linewidth}
				\[
				\begin{tikzcd}
					\Lambda \ar[d,"\pi"'] & T \Lambda \ar[d,"T \pi"] \ar[l] \\
					Y								& T Y \ar[l]
				\end{tikzcd}	
				\]
			\end{column}
		\end{columns}
	\pause
	\begin{defblock}[Tautological $n$-form]
		$\theta \in \Omega^n(\Lambda)$ such that:
		\begin{displaymath}
		\begin{split}
			\left[ \iota_{u_1 \wedge \ldots \wedge u_n} \theta \right]_\eta 
			&= \iota_{(T \pi)_\ast u_1 \wedge \ldots \wedge (T \pi)_\ast u_n} \eta \\
			&= \iota_{u_1 \wedge \ldots \wedge u_n} \pi^\ast \eta 
			\qquad \qquad \forall \eta \in \Lambda \, , \: \forall u_i \in T_\eta \Lambda 		
		\end{split}
		\end{displaymath}
	\end{defblock}
	\vfill
	\begin{columns}
		\begin{column}{.6\linewidth}
			\begin{defblock}[Tautological (multisymplectic) (n+1)-form]
				$$\omega := d \theta$$
			\end{defblock}
		\end{column}
		\begin{column}{.4\linewidth}
		 	\begin{claimblock}$\omega$ is not degenerate.\end{claimblock}	
		\end{column}
	\end{columns}	
	\pause
	\begin{keywordblock}
		\begin{tabular}{|c|c|c|}
			\hline 
			point-particles mechanics & $\rightsquigarrow$ & classical fields mechanics \\
			%(finite discrete DOF) & & (finite dimensional continuous DOF) \\
			\hline 
			symplectic & $\rightsquigarrow$ & multisymplectic \\ 
			\hline 
			Observables (Poisson) algebra & $\rightsquigarrow$ & Observables $L-\infty$ algebra
			 \\ 
			\hline 
			Co-moment map & $\rightsquigarrow$ & Homotopy co-momentum map \\ 
			\hline 
		\end{tabular} 
	\end{keywordblock}

	
\end{frame}
\note[itemize]{
	\item This example is significant from the perspective of geometric classical field theory:
		\begin{displaymath}
			\frac{\text{classical mechanics}}{\text{symplectic geo.}} =
			\frac{\text{classical field mechanics}}{\text{multisymplectic geo.}}
		\end{displaymath}
	\item Multicotangent bundle is the \emph{Higher analogue} of the cotangent bundle.
	(but it is not yet the analogue of a \emph{phase space}.)
\item The multiphase space is the sub-bundle of $n$-forms vanishing when contracted with 2 vertical fields.
  	\item The reason why this sub-bundle has a particular role is that it can be proved to be isomorphic to a suitable dual of the first Jet bundle.
  	\item For further details see Gotay et al. \href{https://arxiv.org/abs/physics/9801019}{arXiv:physics/9801019}. For a pictorial representation of all the structures involved in the geometric mechanics of I order classical field theories see appendix, pag: \ref{frame:Gimmsy}.
}
%------------------------------------------------------------------------------------------------	
	

\begin{frame}[fragile,shrink]{Unwrapping the \emph{higher Jacobi equations}}\label{Frame:unwapping-Jacobi}
\underline{Slogan:} \emph{Jacobi identity satisfied up to an higher coherent homotopy}
		%
		\vspace{1.5em}
		\begin{columns}[c]
			\hfill
			\begin{column}{0.5\linewidth}
				Higher Jacobi implies:
				\begin{itemize}  \setlength\itemsep{1em}
					\item Underlying chain-complex $(L,\mu_1)$ with differential $d=\mu_1$;
					\item \color{red} $\mu_2 = [\cdot,\cdot]$ is a chain map $L^{\otimes 2} \to L$;
					\item \color{green!20!black}$\mu_3=j(\cdot,\cdot,\cdot)$ is a chain homotopy 
						$\mu_2\circ\mu_2 \Rightarrow 0$;
						\\ i.e. between the usual Jacobiator ${[[\cdot,\cdot],\cdot]} \circ P_{\text{unsh}}$ and the $0$ map 
					\item \color{purple}higher analogues...	
					\\ e.g. $\mu_4$, is a second order chain-homotopy between the two chain homotopies  ${[j(\cdot,\cdot ,\cdot]),\cdot]}\circ P_{\text{unsh}}$ and ${j([\cdot , \cdot],\cdot,\cdot)}\circ P_{\text{unsh}}$
				\end{itemize}
			\end{column}
			\begin{column}{0.45\linewidth}
				\includestandalone[width=0.9\linewidth]{Pictures/Figure_Linfinity_diagram}
			\end{column}	
		\end{columns}	
		\vspace{1.5em}
		Notation: $P_{\text{unsh}}$ = sum on all the possibile unshuffled permutation.

\end{frame}
\note[itemize]{
  \item Regarding any $l_k$ as a tree with $k$ entries and 1 output, the $k$-th generalized Jacobi equation is produced summing all the possible way to obtain a $k+1$-ary tree by composing two other trees (not more then two!).
  \item Can be regarded as
  	\begin{displaymath}
  		\sum_{i+j = k} l_j \circ ( l_j \otimes \mathbb{I}) \circ P_{\text{unsh}}
  	\end{displaymath}
  	Where $P_{\text{unsh}} : L^{\otimes(k-1)} \rightarrow L^{\otimes(k-1)} $ is the $(i,j)$-unshuffolator.
  	\\(you consider only unshuffles to avoid the redundancies given by the fact that any $l_i$ has fixed symmetry.
  \item Examples of unshuffles: \\
  \begin{displaymath}
  \begin{split}
  	(12)(3)\quad(13)(2)\quad(23)(1)\\
  	(123)(4)\quad(234)(1)\quad(134)(2)\quad(124)(3)\\
  	(12)(34)\quad(23)(14)\quad(13)(24)\quad(14)(23)\quad(24)(13)
  \end{split}
  \end{displaymath}
	\item When regarding the L$\infty$ structure as a chain complex with homotopies you get a neat intepretation of the Jacobi identity at the price that \emph{graded skew-symmetry} definition is more obscure than in the presentation with graded vector spaces.
}
%------------------------------------------------------------------------------------------------




%------------------------------------------------------------------------------------------------
\end{document}
