%+----------------------------------------------------------------------------+
%| SLIDES: 
%| Chapter: Complementary material - details on eventual questions
%| Author: Antonio miti
%| Event: PHD preliminary Defence
%+----------------------------------------------------------------------------+

%- HandOut Flag -----------------------------------------------------------------------------------------
\newif\ifHandout

%- D0cum3nt ----------------------------------------------------------------------------------------------
\documentclass[beamer,10pt]{standalone}   
%\documentclass[beamer,10pt,handout]{standalone}  \Handouttrue  

%- HandOut Flag -----------------------------------------------------------------------------------------
\ifHandout
	\setbeameroption{show notes} %print notes   
\fi

	
%- Packages ----------------------------------------------------------------------------------------------
\usepackage{custom-style}

%--Beamer Style-----------------------------------------------------------------------------------------------
\usetheme{toninus}



\providecommand{\blank}{\text{\textvisiblespace}}


\newcommand{\subsectiontitle}{
  \begin{frame}
  \vfill
  \centering
  \begin{beamercolorbox}[sep=8pt,center,shadow=true,rounded=true]{title}
    \usebeamerfont{title}\insertsectionhead\par%
    \usebeamerfont{title}\insertsubsectionhead\par%
  \end{beamercolorbox}
  \vfill
  \end{frame}
}

\providecommand{\blank}{\text{\textvisiblespace}}




%---------------------------------------------------------------------------------------------------------------------------------------------------
%- D0cum3nt ----------------------------------------------------------------------------------------------------------------------------------
\begin{document}
%------------------------------------------------------------------------------------------------

%##################################################################################
\begin{frame}
	\begin{center}
	\Huge\emph{Supplementary Material}
	\end{center}
\end{frame}
\note[itemize]{
	\item
}
\addtocounter{framenumber}{-1}
%##################################################################################



%------------------------------------------------------------------------------------------------
\begin{frame}[fragile]{MS geometry and classical field mechanics}\label{Frame:Ms-Field-Mechanics}
		Consider a smooth manifold $Y$,
		\begin{columns}
			\hfill
			\begin{column}{.5\linewidth}
				\emph{Multicotangent bundle} $\bigwedge = \bigwedge^n T^\ast Y$\\
				is naturally $n$-plectic
			\end{column}
			\begin{column}{.4\linewidth}
				\[
				\begin{tikzcd}
					\Lambda \ar[d,"\pi"'] & T \Lambda \ar[d,"T \pi"] \ar[l] \\
					Y								& T Y \ar[l]
				\end{tikzcd}	
				\]
			\end{column}
		\end{columns}
	\pause
	\begin{defblock}[Tautological $n$-form]
		$\theta \in \Omega^n(\Lambda)$ such that:
		\begin{displaymath}
		\begin{split}
			\left[ \iota_{u_1 \wedge \ldots \wedge u_n} \theta \right]_\eta 
			&= \iota_{(T \pi)_\ast u_1 \wedge \ldots \wedge (T \pi)_\ast u_n} \eta \\
			&= \iota_{u_1 \wedge \ldots \wedge u_n} \pi^\ast \eta 
			\qquad \qquad \forall \eta \in \Lambda \, , \: \forall u_i \in T_\eta \Lambda 		
		\end{split}
		\end{displaymath}
	\end{defblock}
	\vfill
	\begin{columns}
		\begin{column}{.6\linewidth}
			\begin{defblock}[Tautological (multisymplectic) (n+1)-form]
				$$\omega := d \theta$$
			\end{defblock}
		\end{column}
		\begin{column}{.4\linewidth}
		 	\begin{claimblock}$\omega$ is not degenerate.\end{claimblock}	
		\end{column}
	\end{columns}	
	\pause
	\begin{keywordblock}
		\begin{tabular}{|c|c|c|}
			\hline 
			point-particles mechanics & $\rightsquigarrow$ & classical fields mechanics \\
			%(finite discrete DOF) & & (finite dimensional continuous DOF) \\
			\hline 
			symplectic & $\rightsquigarrow$ & multisymplectic \\ 
			\hline 
			Observables (Poisson) algebra & $\rightsquigarrow$ & Observables $L-\infty$ algebra
			 \\ 
			\hline 
			Co-moment map & $\rightsquigarrow$ & Homotopy co-momentum map \\ 
			\hline 
		\end{tabular} 
	\end{keywordblock}

	
\end{frame}
\note[itemize]{
	\item This example is significant from the perspective of geometric classical field theory:
		\begin{displaymath}
			\frac{\text{classical mechanics}}{\text{symplectic geo.}} =
			\frac{\text{classical field mechanics}}{\text{multisymplectic geo.}}
		\end{displaymath}
	\item Multicotangent bundle is the \emph{Higher analogue} of the cotangent bundle.
	(but it is not yet the analogue of a \emph{phase space}.)
\item The multiphase space is the sub-bundle of $n$-forms vanishing when contracted with 2 vertical fields.
  	\item The reason why this sub-bundle has a particular role is that it can be proved to be isomorphic to a suitable dual of the first Jet bundle.
  	\item For further details see Gotay et al. \href{https://arxiv.org/abs/physics/9801019}{arXiv:physics/9801019}. For a pictorial representation of all the structures involved in the geometric mechanics of I order classical field theories see appendix, pag: \ref{frame:Gimmsy}.
}
%------------------------------------------------------------------------------------------------	
	

\begin{frame}[fragile,shrink]{Unwrapping the \emph{higher Jacobi equations}}\label{Frame:unwapping-Jacobi}
\underline{Slogan:} \emph{Jacobi identity satisfied up to an higher coherent homotopy}
		%
		\vspace{1.5em}
		\begin{columns}[c]
			\hfill
			\begin{column}{0.5\linewidth}
				Higher Jacobi implies:
				\begin{itemize}  \setlength\itemsep{1em}
					\item Underlying chain-complex $(L,\mu_1)$ with differential $d=\mu_1$;
					\item \color{red} $\mu_2 = [\cdot,\cdot]$ is a chain map $L^{\otimes 2} \to L$;
					\item \color{green!20!black}$\mu_3=j(\cdot,\cdot,\cdot)$ is a chain homotopy 
						$\mu_2\circ\mu_2 \Rightarrow 0$;
						\\ i.e. between the usual Jacobiator ${[[\cdot,\cdot],\cdot]} \circ P_{\text{unsh}}$ and the $0$ map 
					\item \color{purple}higher analogues...	
					\\ e.g. $\mu_4$, is a second order chain-homotopy between the two chain homotopies  ${[j(\cdot,\cdot ,\cdot]),\cdot]}\circ P_{\text{unsh}}$ and ${j([\cdot , \cdot],\cdot,\cdot)}\circ P_{\text{unsh}}$
				\end{itemize}
			\end{column}
			\begin{column}{0.45\linewidth}
				\includestandalone[width=0.9\linewidth]{Pictures/Figure_Linfinity_diagram}
			\end{column}	
		\end{columns}	
		\vspace{1.5em}
		Notation: $P_{\text{unsh}}$ = sum on all the possibile unshuffled permutation.

\end{frame}
\note[itemize]{
  \item Regarding any $l_k$ as a tree with $k$ entries and 1 output, the $k$-th generalized Jacobi equation is produced summing all the possible way to obtain a $k+1$-ary tree by composing two other trees (not more then two!).
  \item Can be regarded as
  	\begin{displaymath}
  		\sum_{i+j = k} l_j \circ ( l_j \otimes \mathbb{I}) \circ P_{\text{unsh}}
  	\end{displaymath}
  	Where $P_{\text{unsh}} : L^{\otimes(k-1)} \rightarrow L^{\otimes(k-1)} $ is the $(i,j)$-unshuffolator.
  	\\(you consider only unshuffles to avoid the redundancies given by the fact that any $l_i$ has fixed symmetry.
  \item Examples of unshuffles: \\
  \begin{displaymath}
  \begin{split}
  	(12)(3)\quad(13)(2)\quad(23)(1)\\
  	(123)(4)\quad(234)(1)\quad(134)(2)\quad(124)(3)\\
  	(12)(34)\quad(23)(14)\quad(13)(24)\quad(14)(23)\quad(24)(13)
  \end{split}
  \end{displaymath}
	\item When regarding the L$\infty$ structure as a chain complex with homotopies you get a neat intepretation of the Jacobi identity at the price that \emph{graded skew-symmetry} definition is more obscure than in the presentation with graded vector spaces.
}
%------------------------------------------------------------------------------------------------



%-------------------------------------------------------------------------------------------------------------------------------------------------
\begin{frame}[t]{Symmetries in \textbf{multisymplectic geometry}}
	Consider a Lie algebra action $v:\mathfrak{g} \to \mathfrak{X}(M)$  preserving the $n$-plectic form $\omega$,
	\vfill

	\vspace{-1em}
	\begin{columns}[T]
		\setlength{\belowdisplayskip}{5pt}
		\begin{column}{.50\linewidth}
			%
			\centering \it
			\onslide<2->{
				$-$ symplectic case $-$
				\begin{defblock}[Comoment map pertaining to $v$]
					Lie algebra morphism
					$$ f: \mathfrak{g} \to C^\infty(M) $$
					such that
					$$ d~f (x) = -\iota_{v_x} \omega \qquad \forall x \in \mathfrak{g}~.$$
				\end{defblock}
			}
		\end{column}	
		%
		\onslide<2->{\vrule{}}
		%
		\begin{column}{.50\linewidth}
			\centering \it
			\onslide<3->{			
				$-$ $n$-plectic case $-$
				\begin{defblock}[Homotopy comoment map \tiny (HCMM)]
					$L_\infty$-morphism 
					$$ (f_k) : \mathfrak{g} \to L_\infty (M,\omega)$$
					such that
					$$ d~f_1(x) = -\iota_{v_x} \omega \qquad \forall x \in \mathfrak{g}~.$$
				\end{defblock}	
			}
		\end{column}	
	\end{columns}	
	%
	\pause
	\vfill
	\centering 
	\onslide<4->{\textbf{-- Conserved quantities --}}
	%
	\vspace{-.5em}
	\begin{columns}[T]
		\setlength{\belowdisplayskip}{5pt}
		\begin{column}{.50\linewidth}
			%
			\centering \it
			\onslide<4->{
			\begin{propblock}[Noether Theorem]
				\small Fixed $H\in C^\infty_{\text{Ham}}(M)$ ($\mathfrak{g}$-invariant) ,
				$$\mathcal{L}_{v_H} f(x) = 0 \qquad \forall x \in \mathfrak{g}$$
			\end{propblock}
			}
		\end{column}	
		%
		\onslide<5->{\vrule{}}
		%
		\begin{column}{.50\linewidth}
			\centering \it
			\onslide<5->{			
			\begin{propblock}[RWZ16 Theorem]
				\small Fixed $H\in \Omega^{n-1}_{\text{Ham}}(M)$ ($\mathfrak{g}$-invariant),
				$$\mathcal{L}_{v_H} f_k(p) \in B^k(M) \qquad \forall p \in Z_k(\mathfrak{g})$$			
			\end{propblock}
			}
		\end{column}	
	\end{columns}		
\end{frame}
\note{
}
%-------------------------------------------------------------------------------------------------------------------------------------------------


%-------------------------------------------------------------------------------------------------------------------------------------------------
\subsection{Homotopy comomentum maps}\label{frame:hcmm-main}
\begin{frame}[fragile]{Homotopy comomentum maps}
	Consider a Lie algebra action $v:\mathfrak{g} \to \mathfrak{X}(M)$  \underline{preserving the $n$-plectic form $\omega$}.
	\vfill
	\begin{defblock}[Homotopy comomentum map \emph{(Callies, Fregier, Rogers, Zambon)}]
		\ifHandout
			\includestandalone{Pictures/Figure_Lifting}
		\else
			\includestandalone{Pictures/Frame_Lifting}
		\fi					
	\end{defblock}
	\onslide<4->{
	\begin{lemblock}[HCMM unfolded  (CFRZ16)]
			%
			HCMM is a sequence of (graded-skew) multilinear maps:
			\begin{displaymath}
				(f)  = \big\lbrace f_k: \; \Lambda^k{\mathfrak g} \to L^{1-k} \subseteq \Omega^{n-k}(M) 
				~\big\vert~ 0\leq k \leq n+1  \big\rbrace
			\end{displaymath}
			\emph{fulfilling:}%\emph{such that:}
			\begin{itemize}
				\item<5-> $f_0 = 0 $, $f_{n+1} = 0$
				\item<6-> $d f_k (p) = f_{k-1} (
				\tikz[baseline,remember picture]{\node[rounded corners,
                        fill=green!5,draw=green!30,anchor=base]            
            			(target) {$\partial $ };
            	}				
				p)  - (-1)^{\frac{k(k+1)}{2}} \iota(v_p) \omega 
				\qquad\scriptstyle \forall p \in \Lambda^k(\mathfrak{g}),\; \forall k=1,\dots n+1$
			\end{itemize}
		\onslide<7->{
			\tikz[overlay,remember picture]
			{
				\node[rounded corners,
	                 draw=green!30,anchor=base]            
	            	 (base) at ($(current page.east)-(3,3)$) [rotate=-0,align=center] {\footnotesize{\hyperlink{frame:CE-complex}{\emph{Chevalley-Eilenberg boundary op.}}}};
			}	
		\begin{tikzpicture}[overlay,remember picture]
	    	\path[->] (base.west) edge[bend right,green](target.north east);
	    \end{tikzpicture}
	    }
	\end{lemblock}	
	}
	\vfill
\end{frame}
\note[itemize]{
	\item  An infinitesimal symmetry is a lie algebra morphism such that $\mathcal{L}_{v_x} \omega = 0 ~ \forall x \in \mathfrak{g}$.
	\\ (It is also call an infinitesimal multisymplectic action and $v_x$ is the infinitesimal generator of the action, corresponding to $x \in \mathfrak g$.) 
	\item Essentially, admitting a comoment maps mean that $v$ acts via Hamiltonian vector fields.
	\item In mechanical terms, a moment map is a tool associated with a Hamiltonian action of a Lie group on a symplectic manifold, used to construct conserved quantities for the action.(see \ref{frame:HCMMandConserved} in appendix.
}
%-------------------------------------------------------------------------------------------------------------------------------------------------

%-------------------------------------------------------------------------------------------------------------------------------------------------
\begin{frame}[fragile,shrink]{Homotopy co-moment maps \emph{(Callies, Fregier, Rogers, Zambon)}}
	\begin{columns}
		\begin{column}{.625\linewidth}	
			HCMM is an $L_\infty$-morphism  $\quad(f):\mathfrak{g}\to L_\infty(M,\omega)$
			\\[.5em]
			 lifting the  infinitesimal action $\quad v:\mathfrak{g}\to \mathfrak{X}(M)$
		\end{column}
		\begin{column}{.325\linewidth}	
			\begin{displaymath}
				\begin{tikzcd}[column sep = large]
					& L_{\infty}(M,\omega) \ar[d,"\mathscr{v}"]
					\\
					\mathfrak{g} \ar[ur,dashed,"(f)"]\ar[r,"v"']& \mathfrak{X}(M)
				\end{tikzcd}	
			\end{displaymath}		
		\end{column}
	\end{columns}
	\pause
	\begin{lemblock}[HCMM unfolded  \cite{Callies2016}]
			%
			HCMM is a sequence of (graded-skew) multilinear maps:
			\begin{displaymath}
				(f)  = \big\lbrace f_k: \; \Lambda^k{\mathfrak g} \to L_{k-1} \subseteq \Omega^{n-k} 
				\;\big\vert\; 0\leq k \leq n+1  \big\rbrace
			\end{displaymath}
			%
			\vspace{-.5em}	
			\includestandalone[width=0.9\textwidth]{Pictures/Frame_HCMM}
			
			\vspace{-1em}		
			\emph{fulfilling:}%\emph{such that:}
			\begin{itemize}
				\item<2-> $f_0 = 0 $, $f_{n+1} = 0$
				\item<3-> $d f_k (p) = f_{k-1} (\partial p)  - (-1)^{\frac{k(k+1)}{2}} \iota(v_p) \omega 
				\qquad\scriptstyle \forall p \in \Lambda^k(\mathfrak{g}),\; \forall k=1,\dots n+1$
			\end{itemize}
		\end{lemblock}

	\begin{tamblock}
	 Practically a HCMM is given by several multilinear maps
	 \begin{displaymath}
	 	f_i = \Lambda^i \mathfrak{g} \to L_{i-1}
	 \end{displaymath}
	 satisfying:
	 \begin{enumerate}
	 	\item $ d f_1(\xi) = - \iota_{v_\xi} \omega$
	 	\item $\sum ...$
	 \end{enumerate}
	\end{tamblock}


\end{frame}
\note[itemize]{
	%\item 		Consider:  $v:\mathfrak g\to \mathfrak X(M)$  a Lie algebra morphism  s.t. $\mathcal{L}_{v_x}\omega=0 \quad  \forall x\in\mathfrak g$ (i.e infinitesimal multisymplectic Lie algebra action $\mathfrak{g}\circlearrowleft (M,\omega)$)
	\item More conceptually, a comoment is an $L_\infty$-morphism $(f):\mathfrak{g}\to L_\infty(M,\omega)$ lifting the action $v:\mathfrak{g}\to \mathfrak{X}(M)$, 
i.e. making the diagram commute in the $L_\infty$-algebras category.
	\item The vertical arrow is the trivial $L_\infty$-extension of the function mapping any Hamiltonian form to the unique corresponding Hamiltonian vector field (an it is zero elsewhere)
		\\
		(Note that any Lie algebra can be seen as an $L_\infty$-algebra concentrated in degree $0$, therefore any $L_\infty$-morphism $L\to\mathfrak{g}$ is simply given by a linear map $L_0 \to \mathfrak{g}$ preserving the binary brackets.)
	\item We will make use of an explicit version of this definition which is expressed by the lemma.
	 Practically speaking, a HCMM is given by several multilinear maps ...
	 \item In the equation we have tacitly set $\Lambda^{-1}(M) = 0$
	 %\item Notation: \qquad $\partial =$ Chevalley-Eilenberg boundary operator.
	%\item Notice that a HCMM pertains to an "infinitesimal" action of ${\mathfrak g}$ on $M$ with ${\mathfrak g}$ being the Lie algebra of a generic Lie group $G$, acting on $M$ by $\omega$-preserving vector fields.
		\item (Notation) $ p = \xi_1 \wedge \xi_2 \wedge \dots \wedge \xi_k$, 
			then $v_p = v_1 \wedge v_2 \wedge \dots \wedge v_k$ 
			where $v_i \equiv v_{\xi_i}$ are the fundamental vector fields associated to the action $G \circlearrowright M$.
	%	\item (Notation) $\iota(v_p) \omega = \iota(v_k)\dots\iota(v_1) \omega$
	%	\item $\varsigma(k) := - (-1)^{\frac{k(k+1)}{2}}$ 
		\item (Notation) $(\iota^{k}_{\mathfrak{g}}\omega)(p):= \iota(v_p) \omega = \iota(v_k)\dots\iota(v_1) \omega$
		\item $\partial \equiv \partial_k:  \Lambda^{k} {\mathfrak g} \to \Lambda^{k-1} {\mathfrak g}$  is the usual Eilenberg-Chevalley complex boundary operator (see appendix, pag: \ref{frame:CE-complex});
%		\item The definition tells us that the {\it closed} forms
%			$$\mu_k := f_{k-1} (\partial p) +  \varsigma(k) \iota(v_p) \omega 	$$
%			must actually be {\it exact}, with potential $-f_k(p)$.  	
		\item The last equation tells us that an HCMM is not a chain complex morphism but is rather a chain complex homotopy between 0 and the multicontraction $\alpha=(\iota^{k}_{\mathfrak{g}}\omega)$ (see next slide).
		is a chain map by lemma 2.18 \cite{Ryvkin2016}).
}
%---------------------------------------------------------------------------------------------------------------------------------



%------------------------------------------------------------------------------------------------
\end{document}
